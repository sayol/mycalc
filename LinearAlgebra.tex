\documentclass[12pt,a4paper,fleqn]{mycalc}
\everymath{\color{mycalc}}
\begin{document}
	\chapter{Matrix Analysis}
	\section{Introduction}
	\section{Review of Matrix Algebra}
	\subsection{Definition}
	\begin{enumerate}[a]
		\item An array of real numbers 
		\(
		A=\begin{bmatrix}
		a_{11} & a_{12} & a_{13} & \cdots & a_{1n}\\
		a_{21} & a_{22} & a_{23} & \cdots & a_{2n}\\
		\vdots & \vdots & \vdots & \vdots & \vdots\\
		a_{m1} & a_{m2} & a_{m3} & \cdots & a_{mn}\\
		\end{bmatrix}
		\)
		is called \emph{\( m\times n \) matrix}. The \( a_{ij} \) is referred to as the \( i \), \( j \)-th element and denotes the element in the \( i \)-th row and \( j \)-th column. If \( m = n \) then \( A \) is called a \emph{square matrix of order \( n \)}. If the matrix has one column or one row then it is called a \emph{column vector} or a \emph{row vector} respectively.
		\item In a square matrix A of order n the diagonal containing the elements \( a_{11}, a_{22}, \dots ,
		a_{nn} \) is called the principal or \emph{leading diagonal}. The sum of the elements in this diagonal is called the \emph{trace of \( A \)}, that is \( \operatorname{trace} A=\sum_{i=1}^{n}a_{ii}. \)
		\item A \emph{diagonal matrix} is a square matrix that has its only non-zero elements along the leading diagonal. A special case of a diagonal matrix is the \emph{unit} or \emph{identity matrix} \( I \) for which \( a_{11} = a_{22} = \cdots = a_{nn} = 1. \)
		\item A \emph{zero} or \emph{null matrix} \( 0 \) is a matrix with every element zero.
		\item The \emph{transposed matrix} \( A^{T} \) is the matrix \( A \) with rows and columns interchanged,
		its \( i \), \( j \)-th element being \( a_{ji}. \)
		\item A square matrix \( A \) is called a \emph{symmetric matrix} if \( A^{T}=A. \) It is called \emph{skew symmetric} if \( A^{T}=-A. \)
	\end{enumerate}
	\subsection{Basic Operations on Matrices}
	\subsubsection{Equality}
	The matrices \( A \) and \( B \) are equal, that is \( A = B \), if they are of the same order \( m \times n \) and \( a_{ij}=b_{ij},\quad 1\leq i\leq m,\quad 1\leq j\leq n. \)
	\subsubsection{Mutiplication by A Scalar}
	If \( \lambda \) is a scalar then the matrix \( \lambda A \) has elements \( \lambda a_{ij}. \)
	\subsubsection{Addition}
	We can only add an \( m \times n \) matrix \( A \) to another \( m \times n \) matrix \( B \) and the elements of the sum \( A + B \) are
	\( a_{ij} + b_{ij},\quad 1 \leq i \leq m,\quad 1 \leq j \leq n. \)
	\begin{tcolorbox}[title={Properties of Addition}]
		\begin{enumerate}[i]
			\item commutative law: \( A + B = B + A \)
			\item associative law: \( (A + B) + C = A + (B + C) \)
			\item distributive law: \( \lambda (A + B) = \lambda A + \lambda B, \lambda \) scalar.
		\end{enumerate}
	\end{tcolorbox}
	\subsubsection{Matrix multiplication}
	If \( A \) is an \( m \times p \) matrix and \( B \) a \( p \times n \) matrix then we define the product \( C = AB \) as the
	\( m \times n \) matrix with elements
	\begin{align*}
	c_{ij}=\sum_{k=1}^{p}a_{ik}b_{kj},\quad i=1,2,\dots,m;\quad j=1,2,\dots,n.
	\end{align*}
	\begin{tcolorbox}[title={Properties of multiplication}]
		\begin{enumerate}[i]
			\item The commutative law is not satisfied in general; that is, in general \( AB \neq BA. \) Order matters and we distinguish between \( AB \) and \( BA \) by the terminology: pre-multiplication of \( B \) by \( A \) to form \( AB \) and post-multiplication of \( B \) by \( A \) to form \( BA. \)
			\item Associative law: \( A(BC) = (AB)C \)
			If \( \lambda  \) is a scalar then \( (\lambda A)B = A(\lambda B) = \lambda AB \)
			\item Distributive law over addition:
			\( (A + B)C = AC + BC\\
			A(B + C) = AB + AC \)\\
			Note the importance of maintaining order of multiplication.
			\item If \( A \) is an \( m \times n \) matrix and if \( I_m \) and \( I_n \) are the unit matrices of order \( m \) and \( n \) respectively then \( I_mA=AI_n=A. \)
		\end{enumerate}
	\end{tcolorbox}
	\begin{tcolorbox}[title={Properties of the Transpose}]
		If \( A^{T} \) is the transposed matrix of \( A \) then
		\begin{enumerate}[i]
			\item \( (A + B)^{T} = A^{T} + B^{T} \)
			\item \( (A^{T})^{T} = A \)
			\item \( (AB)^{T} = B^{T}A^{T} \)
		\end{enumerate}
	\end{tcolorbox}
	\subsection{Determinants}
	The determinant of a square \( n \times n \) matrix \( A \) is denoted by det \( A \) or \( | A |. \) If we take a determinant and delete row \( i \) and column \( j \) then the determinant remaining is called the minor \( M_{ij} \) of the \( i \), \( j \)-th element. In general we can take any row \( i \) (or column) and evaluate an \( n \times n \) determinant \( | A | \) as \( |A|=\sum_{j=1}^{n}(-1)^{i+j}a_{ij}M_{ij}. \)\\
	A minor multiplied by the appropriate sign is called the \emph{cofactor} \( A_{ij} \) of the \( i \) , \( j \)-th element so \( A_{ij} = (-1)^{i+j} M_{ij} \) and thus \( |A|=\sum_{j=1}^{n}a_{ij}A_{ij}. \)
	\begin{tcolorbox}[title={Some useful properties}]
		\begin{enumerate}[i]
			\item \( |A^{T}|=|A| \)
			\item \( |AB|=|A||B| \)
			\item A square matrix \( A \) is said to be non-singular if \( | A | \neq 0 \) and singular if \( | A | = 0. \)
		\end{enumerate}
	\end{tcolorbox}
	\subsection{Adjoint and inverse matrices}
	\subsubsection{Adjoint matrix}
	The \emph{adjoint} of a square matrix \( A \) is the transpose of the matrix of cofactors, so for a \( 3 \times 3 \) matrix \( A \)
	\begin{align*}
	\operatorname{adj} A=\begin{bmatrix}
	A_{11} & A_{12} & A_{13}\\
	A_{21} & A_{22} & A_{23}\\
	A_{31} & A_{32} & A_{33}
	\end{bmatrix}^{T}
	\end{align*}
	\begin{tcolorbox}[title={Properties}]
		\begin{enumerate}[i]
			\item If \( A \) is non-singular then \( |A | \neq 0 \) and \( A^{-1} = (\operatorname{adj} A)/|A |. \)
			\item If \( A \) is singular then \( |A | = 0 \) and \( A^{-1} \) does not exist.
			\item \( (AB)^{-1} = B^{-1}A^{-1}. \)
		\end{enumerate}
	\end{tcolorbox}
	\begin{tcolorbox}[title={Matlab Practice}]
		Do matrix operations in \emph{Matlab}.
	\end{tcolorbox}
	\subsection{Linear aligns}
	In this section we reiterate some definitive statements about the solution of the \emph{system of simultaneous linear aligns}
	\begin{align*}
	a_{11}x_1+a_{12}x_2+\cdots+a_{1n}x_n &=b_1\\
	a_{21}x_1+a_{22}x_2+\cdots+a_{2n}x_n &=b_1\\
	&\hphantom{=} \vdots\\
	a_{n1}x_1+a_{n2}x_2+\cdots+a_{nn}x_n &=b_1
	\end{align*}
	or, in matrix notation,
	\begin{align*}
	\begin{bmatrix}
	a_{11} & a_{12} & \cdots & a_{1n}\\
	a_{21} & a_{22} & \cdots & a_{2n}\\
	\vdots & \vdots & \vdots & \vdots\\
	a_{n1} & a_{n2} & \cdots & a_{nn}\\
	\end{bmatrix}
	\begin{bmatrix}
	x_1\\
	x_2\\
	\vdots\\
	x_n
	\end{bmatrix}
	=
	\begin{bmatrix}
	b_1\\
	b_2\\
	\vdots\\
	b_n
	\end{bmatrix}
	\end{align*}
	that is,
	\begin{align*}
	Ax=b
	\end{align*}
	where \( A \) is the matrix of coefficients and \( x \) is the vector of unknowns. If \( b = 0 \) the aligns are called \emph{homogeneous}, while if \( b \neq 0 \) they are called \emph{nonhomogeneous} (or
	inhomogeneous). Considering individual cases:
	\begin{enumerate}[label={\protect\color{mycalc}Case~(\roman*)}]
		\item If \( b \neq 0 \) and \( |A | \neq 0 \) then we have a unique solution \( x = A^{-1}b. \)
		\item If \( b = 0 \) and \( |A | \neq 0 \) we have the trivial solution \( x = 0. \)
		\item If \( b \neq 0 \) and \( |A | = 0 \) then we have two possibilities: either the aligns are inconsistent
		and we have no solution or we have infinitely many solutions.
		\item If \( b = 0 \) and \( |A | = 0 \) then we have infinitely many solutions.
		\item[] Case (iv) is one of the most important, since from it we can deduce the important result that the homogeneous align \( Ax = 0 \) has a non-trivial solution if and only if \( |A | = 0. \)
	\end{enumerate}
	\begin{tcolorbox}[title={Matlab Practice}]
		Solve system of simultaneous linear equations.
	\end{tcolorbox}
	\subsection{Rank of a Matrix}
	\begin{tcolorbox}
		If \( A \) and \( (A : b) \) have different rank then we have no solution to \( Ax+b. \) If the two matrices have the same rank then a solution exists, and furthermore the solution will contain a number of free parameters equal to \( (n - \operatorname{rank} A). \)
	\end{tcolorbox}
	\section{Vector Spaces}
	\subsection{Definition}
	A real vector space V is a set of objects called vectors together with rules for addition and multiplication by real numbers. For any three vectors \( a, b \) and \( c \) in \( V \) and any real numbers \(\alpha\) and \(\beta\) the sum \( a + b \) and the product \(\alpha\) a also belong to \( V \) and satisfy the following axioms:
	\begin{enumerate}[a]
		\item \( a+b=b+a \)
		\item \( a+(b+c)=(a+b)+c \)
		\item there exists a zero vector 0 such that \( a+0=a \)
		\item or each \( a \) in \( V \) there is an element \( -a \) in \( V \) such that \( a+(-a)=0 \)
		\item \( \alpha (a + b) = \alpha a + \alpha b \)
		\item \( (\alpha + \beta)a = \alpha a + \beta a \)
	\end{enumerate}
	\subsection{Linear independence}
	The idea of linear dependence is a general one for any vector space. The vector \( x \) is said to be \emph{linearly dependent} on \( x_1, x_2, \cdots , x_m \) if it can be written as \( x = \alpha_1x_1 + \alpha_2x_2 + \cdots + \alpha_mx_m \) for some scalars \( \alpha_1,\cdots, \alpha_m. \) The set of vectors \( y_1, y_2, \cdots , y_m \) is said to be \emph{linearly independent} if and only if \( \beta_1y_1 + \beta_2y_2 + \cdots + \beta_my_m = 0. \)
	\begin{example}
		Show that \( e_1=\begin{bmatrix}
		1\\
		0\\
		0
		\end{bmatrix} \) and \( e_2=\begin{bmatrix}
		0\\
		1\\
		0
		\end{bmatrix} \) form a linearly independent set and describe \( S(e_1, e_2) \) geometrically.
	\end{example}
	If we can find a set \( B \) of linearly independent vectors \( x_1, x_2, \cdots , x_n \) in \( V \) such that \( S(x_1, x_2, \cdots , x_n) = V \) then \( B \) is called a \emph{basis} of the vector space \( V. \) Such a basis forms a crucial part of the theory, since every vector \( x \) in \( V \) can be written uniquely as \( x = \alpha_1x_1 + \alpha_2x_2 + \cdots + \alpha_nx_n. \) The definition of \( B \) implies that \( x \) must take this form. To establish uniqueness, let us assume that we can also write \( x \) as \( x = \beta_1x_1 + \beta_2x_2 + \cdots + \beta_nx_n. \) Then, on subtracting, \( 0 = (\alpha_1 - \beta_1)x_1 + \cdots + (\alpha_n - \beta_n)x_n \) and since \( x_1, \cdots , x_n \) are linearly independent, the only solution is \( \alpha_1 = \beta_1, \alpha_2 = \beta_2, \cdots ; \) hence the two expressions for x are the same. It can also be shown that any other basis for \( V \) must also contain \( n \) vectors and that any \( n + 1 \) vectors must be linearly dependent. Such a vector space is said to have
	dimension \( n \) (or infinite dimension if no finite \( n \) can be found). In a three-dimensional
	\emph{Euclidean space}
	\begin{align*}
	e_1=\begin{bmatrix}
	1\\
	0\\
	0
	\end{bmatrix},\quad
	e_2=\begin{bmatrix}
	0\\
	1\\
	0
	\end{bmatrix},\quad
	e_3=\begin{bmatrix}
	0\\
	0\\
	1
	\end{bmatrix}
	\end{align*}
	form an obvious basis, and
	\begin{align*}
	d_1=\begin{bmatrix}
	1\\
	0\\
	0
	\end{bmatrix},\quad
	d_2=\begin{bmatrix}
	1\\
	1\\
	0
	\end{bmatrix},\quad
	d_3=\begin{bmatrix}
	1\\
	1\\
	1
	\end{bmatrix}
	\end{align*}
	is also a perfectly good basis. 
	\subsection{Transformations between bases}
	Since any basis of a particular space contains the same number of vectors, we can look at transformations from one basis to another. We shall consider a three-dimensional space, but the results are equally valid in any number of dimensions. Let \( e_1, e_2, e_3 \) and \( e_1', e_2', e_3' \) be two bases of a space. From the definition of a basis, the vectors \( e_1', e_2' \) and \( e_3' \) can be written in terms of \( e_1, e_2 \) and \( e_3 \) as
	\begin{align*}
	\left.
	\begin{matrix}
	e_1'=a_{11}e_1+a_{21}e_2+a_{31}e_3\\
	e_2'=a_{12}e_1+a_{22}e_2+a_{32}e_3\\
	e_3'=a_{13}e_1+a_{23}e_2+a_{33}e_3
	\end{matrix}
	\right\}
	\end{align*}
	Taking a typical vector x in V, which can be written both as
	\begin{align*}
	x=x_{1}e_1+x_{2}e_2+x_{3}e_3
	\end{align*}
	and as
	\begin{align*}
	x=x_{1}'e_1'+x_{2}'e_2'+x_{3}'e_3'
	\end{align*}
	we can use the transformation to give
	\begin{align*}
	x &= x_1'(a_{11}e_1 + a_{21}e_2 + a_{31}e_3) + x_2'(a_{12}e_1 + a_{22}e_2 + a_{32}e_3) + x_3'(a_{13}e_1 + a_{23}e_2 + a_{33}e_3)\\
	&= (x_1'a_{11} + x_2'a_{12} + x_3'a_{13})e_1 + (x_1'a_{21} + x_2'a_{22} + x_3'a_{23})e_2 + (x_1'a_{31} + x_2'a_{32} + x_3'a_{33})e_3
	\end{align*}
	Then
	\begin{align*}
	x_1&=a_{11}x_1'+a_{12}x_2'+a_{13}x_3'\\
	x_2&=a_{21}x_1'+a_{22}x_2'+a_{23}x_3'\\
	x_3&=a_{31}x_1'+a_{32}x_2'+a_{33}x_3'
	\end{align*}
	or \( x=Ax'. \)
	\par
	Thus changing from one basis to another is equivalent to transforming the coordinates by multiplication by a matrix, and we thus have another interpretation of matrices. Successive transformations to a third basis will just give \( x' = Bx'', \) and hence the composite transformation is \( x = (AB)x'' \) and is obtained through the standard matrix rules.
	\par
	For convenience of working it is usual to take mutually orthogonal vectors as a basis, so that \( e_i^{T}e_j=\delta_{ij} \) and \( e_i'^{T}e_j'=\delta_{ij} \), where \(\delta_{ij}\) is the \emph{Kronecker delta}
	\begin{align*}
	d_{ij}=\begin{cases}
	1 &\textnormal{if}\quad i=j\\
	0 &\textnormal{if}\quad i\neq j
	\end{cases}
	\end{align*}
	Multiplying out these orthogonality relations, we have
	\begin{align*}
	e_1'^{T}e_j'=\sum_{k}a_{ki}e_k^{T}\sum_{p}a_{pj}e_p
	=\sum_{k}\sum_{p}a_{ki}a_{pj}e_k^{T}e_p
	=\sum_{k}\sum_{p}a_{ki}a_{pj}\delta_{kp}
	=\sum_{k}a_{ki}a_{kj}
	\end{align*}
	Hence
	\begin{align*}
	\sum_{k}a_{ki}a_{kj}=\delta_{ij}
	\end{align*}
	or in matrix form
	\begin{align*}
	A^{T}A=I
	\end{align*}
	It should be noted that such a matrix \( A \) with \( A^{-1} = A^{T} \) is called an \emph{orthogonal matrix.}
	\section*{Exercises}
	\begin{enumerate}
		\item To be written!
	\end{enumerate}
	\section{The Eigenvalue Problem}
	A problem that leads to a concept of crucial importance in many branches of mathematics and its applications is that of seeking non-trivial solutions \( x \neq 0 \) to the matrix equation
	\begin{align*}
	Ax = \lambda x
	\end{align*}
	This is referred to as the eigenvalue problem; values of the scalar \( \lambda \) for which nontrivial solutions exist are called \emph{eigenvalues} and the corresponding solutions \( x \neq 0 \) are
	called the \emph{eigenvectors.}
	\subsection{The characteristic equation}
	The set of simultaneous equations
	\begin{align*}
	Ax=\lambda x
	\end{align*}
	where \( A \) is an \( n \times n \) matrix and \( x = [x_1 x_2 \cdots x_n]^{T} \) is an \( n \times 1 \) column vector can be written in the form
	\begin{align*}
	(\lambda I-A)x=0
	\end{align*}
	where \( I \) is the identity matrix. We know that a non-trivial solution exists if
	\begin{align*}
	c(\lambda) = |\lambda I - A | = 0
	\end{align*}
	Here \( c(\lambda) \) is the expansion of the determinant and is a polynomial of degree \( n \) in \(\lambda\), called the \emph{characteristic polynomial} of \( A. \) Thus
	\begin{align*}
	c(\lambda)=\lambda^{n}+c_{n-1}x^{n-1}+c_{n-2}x^{n-2}+\cdots+c_1\lambda+c_0
	\end{align*}
	The values of \( \lambda \) are precisely the values that satisfy the characteristic equation, and are called the eigenvalues of \( A. \)
	\begin{example}
		Find the characteristic equation for the matrix
		\( A=\begin{bmatrix*}[r]
		1 & 1 & -2\\
		-1 & 2 & 1\\
		0 & 1 & -1
		\end{bmatrix*}. \)
	\end{example}
	\subsubsection{The method of Faddeev}
	If the characteristic polynomial of an \( n \times n \) matrix \( A \) is written as
	\begin{align*}
	\lambda^n - p_1\lambda^{n-1} - \cdots - p_{n-1}\lambda - p_n
	\end{align*}
	then the coefficients \( p_1, p_2, \cdots , p_n \) can be computed using
	\begin{align*}
	p_r=\operatorname{trace} A,\quad (r=1,2,\cdots,n)
	\end{align*}
	where
	\begin{align*}
	A_r=\begin{cases}
	A &(r=1)\\
	AB_{r-1}&(r=2,3,\cdots,n)
	\end{cases}
	\end{align*}
	and
	\begin{align*}
	B_r=A_r-p_rI,\quad\textnormal{where \( I \) is the \( n\times n \) identity matrix}
	\end{align*}
	The calculations may be checked using the result that
	\begin{align*}
	B_n = A_n - p_nI\quad\textnormal{must be the zero matrix}
	\end{align*}
	\subsection{Eigenvalues and Eigenvectors}
	\begin{example}
		Determine the eigenvalues and eigenvectors for the matrix \( A=\begin{bmatrix*}[r]
		1 & 1 & -2\\
		-1 & 2 & 1\\
		0 & 1 & -1
		\end{bmatrix*}. \)
	\end{example}
	\subsection{Exercises}
	\begin{enumerate}
		\item To be written!
	\end{enumerate}
	\subsection{Repeated Eigenvalues}
	\begin{example}
		Determine the eigenvalues and corresponding eigenvectors of the matrix 
		\( A =\begin{bmatrix*}[r]
		3 & -3 & 2\\
		-1 & 5 & -2\\
		-1 & 3 & 0
		\end{bmatrix*}. \)
	\end{example}
	\begin{solution}
		\( \lambda_1=4,\lambda_2=\lambda_3=2 \)
	\end{solution}
	\section{Numerical Methods}
	\section{Reduction to Canonical Form}
	\section{Functions of a Matric}
	\section{Single Value Decomposition}
	\section{State-space Representation}
	\section{Solution of the State align}
	\section{Engeneering Application: Lyapunov Stability Analysis}
	\section{Engineering Application: Capacitor Microphone}
\end{document}